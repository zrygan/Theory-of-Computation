\section{Problems}

\paragraph{Problems on Context-Free Grammars}
\begin{enumerate}
    \item Determine whether the grammar $G_1 = (V,V,R,S)$ is a \texttt{CFG}.
    \item Determine the grammar $E$ that is able to produce any English sentence that starts with a noun, has a verb in the middle, and ends with a noun. For instance, "He ate a burger"
    \item Recursive productions are possible in a \texttt{CFG}. Provide an example, if possible.
    \item The grammar $G_2$ has a production $p\in R$ such that $p\to pp$, show that $L(G_2)\in$\texttt{ CFL}. If $G_2$ is a \texttt{CFL}, what is the language (set of symbols) of $G_2$.
    \item A sentence produced by a \texttt{CFG} can contain non-terminal symbols.
    \item During the production of a string $s$ by some \texttt{CFG}, it is possible that at some point $s$ will contain a non-terminal symbol.
    \item A grammar can have more than one terminal symbol for a single production.
    \item Show that $\forall M, L(M) = \mathcal{L}$ where $M$ is a \texttt{FSM} and $\mathcal{L}$ is a \texttt{CFL}. Similarly, $\forall G, L(G) \neq \mathcal{K}$ where $G$ is a \texttt{CFG} and $\mathcal{K}$ is a regular language.
    \item Construct the grammar $\mathcal{M}$ that can produce any arithmetic operation of two integers. Recall that a binary arithmetic operation is the  addition, subtraction, multiplication, or division of two numbers (in this case, integers).
\end{enumerate}