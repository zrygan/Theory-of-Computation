\section{Pumping Lemma}
\label{pumping_lemma_chapter}

The \textbf{Pumping Lemma} is a \textit{lemma} that aims to describe \blue{property of all regular languages}. This property is that, for every string $s$ in any regular language $\mathcal{L}$ a middle substring in $s$ is \blue{pumpable}, that is, we can infinitely repeat that middle substring of $s$.

Another way of saying this is that we have $s=s_1 s_2 \dots s_n$ and $\exists s_i\in s(s_i:s_1 \neq s_1 \vee s_1 \neq s_n)$ we can infinitely repeat this substring $s_i$, or $(s_i)^+$ and this new string will still be in $\mathcal{L}$.

Take, for example the regular language of the regular expression \texttt{ab$^*$}. In this example, we can look at the second group in the regular expression \texttt{b$^*$} which means that \texttt{b$^*$} is $\lambda$, \texttt{b}, \texttt{b$\dots$b} where there are an arbitrary number of \texttt{b}. Therefore, the symbol \texttt{b} is pumpable, or we can say that \texttt{ab$^*$} $\equiv$ \texttt{ab$^i$b$^*$}, $\forall i \in \mathbb{Z}^+$.

\paragraph{The Pumping Lemma.} 
\label{pumping_lemma_defn}
Let $\mathcal{L}$ be a regular language. Then, there exists an integer $p \geq 1$ that depends on $\mathcal{L}$ such that $\forall w\in \mathcal{L}$ and $|w| \geq p$, $w=xyz$ are a concatenation of three substrings $w,y,z$ which satisfy the following conditions:

\begin{enumerate}
    \item $|y|\geq  1$ (the length of substring $y$ is at least 1)
    \item $|xy| \leq p$ (the length of the substring $xy$ is at most $p$)
    \item $\forall n\geq 0,xy^nz\in \mathcal{L}$ (for every pumping of $y$ the string is still in $\mathcal{L}$)
\end{enumerate}

\paragraph{Formal Definition of Pumping Lemma}
\label{pumping_lemma_formal}
\begin{center}
    \begin{tabular}{l}
     $\forall \mathcal{L}\subseteq \Sigma^*, \mathcal{L} \text{ is regular} \Longrightarrow$  \\
     $\quad \exists p\geq 1,\forall w\in \mathcal{L},|w|\geq p \Longrightarrow$ \\
     $\quad\quad \exists x,y,z\in\Sigma^*,(w=xyz)\wedge (|y|\geq 1)\wedge (|xy|\leq p)\wedge \left(\forall n\geq 0,xy^nz\in \mathcal{L}\right)$
\end{tabular}
\end{center}

So, the goal of the Pumping lemma is to \textit{ prove} that a language $\mathcal{L}$ is regular. But we can also use this lemma to \textit{disprove} that a language is regular, or to prove that a language is irregular.

\subsection{Irregularity of a Language}.

A language is \textbf{irregular} if and only if no Finite State Automata (\texttt{FSA}), regular expression (\texttt{RE}), or regular grammar (\texttt{RG}) recognizes that language. That is to say, 

\[
    \mathcal{L} \text{ is irregular} \Longleftrightarrow \forall M, L(M)\neq \mathcal{L} \text{ where $M$ is a regular formal system}
\]

So, we can prove that a language is irregular by attempting to construct a regular formal system for it and showing that it is \textit{impossible} to do so. However, we can also use the \textit{contrapositive} of the \hyperref[pumping_lemma_defn]{Pumping lemma}.

\paragraph{Contrapositive of the Pumping Lemma.}

Let $\mathcal{L}$ be any language, if the language $\mathcal{L}$ does not satisfy \hyperref[pumping_lemma_defn]{Pumping lemma} it follows that $\mathcal{L}$ is an irregular language. Therefore, if we can \textit{prove} that $\mathcal{L}$ does not satisfy the Pumping lemma, then we prove that it is irregular.

\begin{ex}
    Show that $\mathcal{L} = \left\{\texttt{a$^n$b$^n$}, n\geq0\right\}$ is irregular where the occurrence of \texttt{a} and \texttt{b} are denoted by $n_a$ and $n_b$ respectively.
\end{ex}

\begin{proof}
    $\mathcal{L}$ is irregular

    \textbf{Step 1: State the Pumping lemma}
    \begin{enumerate}
        \item $w=xyz, |w| \geq p$
        \item $y\geq 1, y\notin\lambda$
        \item $|xy|\leq p$
        \item $\forall i\geq 0, xy^iz\in\mathcal{L}$
    \end{enumerate}
    
    \textbf{Step 2: Assume the contradiction is true}
    
    Assume $\mathcal{L}$ is a regular. 
    
    \noindent\textbf{Step 3: Pick $w$}
    
    Set $w$ to be the string $\texttt{a}^j \texttt{b}^j$ where $j+j=2j \geq p$ to meet criterion (1). 
    
    \textbf{Step 4: Determine $xyz=w$}
    
    From step 3 we need to deconstruct $w$ as a concatenation of three strings $x,y,z$ such that $|xy| \leq p$. Since, $2j\geq p$ we can say that $j=k+l$, then $2n=k+l+j \geq p$. Applying this to $\texttt{a}^jb^j$, we get $\texttt{a}^k \texttt{a}^l \texttt{b}^j$. Finally, we can set the following $x=\texttt{a}^k$, $y=\texttt{a}^l$, and $z=\texttt{b}^j$. 
    
    By doing so, we still meet the criteria (2) $y\notin \lambda$ because $y=\texttt{a}^l$; and (3) $|xy|\leq p$ because $|xy|=|\texttt{a}^k \texttt{a}^l|\leq p$ since $k+l=p$. 
    
    \textbf{Step 5: Determine for all pumps of $y$, $w\in \mathcal{L}$}
    
    Referring to criterion (4), we get $xy^iz = \texttt{a}^k \texttt{a}^{li} \texttt{b}^j$; we simply put the $i$ variable as a string repetition of $y$ and substituted the values of $x,y,z$ we got from step 4. 
    
    Let $i=0$, it follows that $\texttt{a}^k\texttt{a}^{l\times0}\texttt{b}^j$. Recall, $\forall s, s^0=1$ and $s\times1=s$. Since, $\texttt{a}^{l\times0}=a^1=1$, then $\texttt{a}^k\texttt{b}^j$. 
    
    From this, since we know that $j=k+l$ and the constraint $\mathcal{L}$ is that the number of occurrences of \texttt{a} is equal to \texttt{b}. But, for $i=0$ we get the string $\texttt{a}^k\textit{b}^j$, $n_a \neq n_b$ since $n_a = k$ and $n_b=j$. 
    
    This then completes the proof of contradiction, it is impossible for $\mathcal{L}$ to be regular. Therefore, $\mathcal{L}$ is irregular. 
\end{proof}
    
\subsubsection{Proof of Irregularity}

\begin{enumerate}
    \item State the \hyperref[pumping_lemma_defn]{Pumping lemma}
    \item Assume that $\mathcal{L}$ is regular
    \item Create a string where the numerical quantifiers of the language are the pumping length $p$.
    \item Let $w$ be this resulting string. 
    \item Determine all combinations of $w$ that meet criteria (1) and (2) of the Pumping lemma.
    \begin{itemize}
        \item Reducing the complexity: by letting the substring $xy$ be a substring of one symbol $\in \Sigma$ we reduce the number of combinations to one.
    \end{itemize}
    \item Set $i$ to a value that will contradict criterion (3). 
    \begin{itemize}
        \item \textit{Brute-force}: Set $i$ to any valid value and determine if the resulting string $s\in \mathcal{L}$. If it is, repeat. Otherwise, conclude. 
    \end{itemize}
    \item State the conclusion, "therefore $\mathcal{L}$ is irregular".
\end{enumerate}

\begin{ex}
    Show that $\mathcal{L}=\{q^nrq^n|n\in\mathbb{Z^+}\}$ is irregular.
\end{ex}

\begin{proof}
    $\mathcal{L}$ is irregular through proof of contradiction with the Pumping lemma: 
    \begin{enumerate}
        \item $w=xyz, |w| \geq p$
        \item $y\geq 1, y\notin\lambda$
        \item $|xy|\leq p$
        \item $\forall i\geq 0, xy^iz\in\mathcal{L}$
    \end{enumerate}
    
    Assume $\mathcal{L}$ is regular.
    
    Let $n = p$ and let this new string be $w$, then
    
    \[
    w = q^p r q^p
    \]
    
    Since $w=xyz$ we need to determine substrings $x,y,z$ that when concatenated to each other, result in $w$. Furthermore, the length of $x,y$ or $|xy|\leq p$ and $p\notin \lambda$. Then,
    
    \begin{center}
        \begin{tabular}{llll}
             $q^prq^p = $   & $\quad x$   & $\quad y$    & $\quad z$        \\
                            & $\quad q^a$ & $\quad q^b$  & $\quad r q^p$  
        \end{tabular}
    \end{center}
    
    This is the only possible combination of $w$ given the criteria (2). Since if $y$ spans all $q$'s and the $r$, and the number of $q$'s in $w$ are $p$. Then, the substring $xy=q^pr$ has size greater than $p$ because $|xy| =p+q\nleq p$
    
    Set $i$ to be a value that will contradict criterion (3). In this case, let's set $i=2$
    \begin{align*}
        w & = xy^iz & \\
          & = q^aq^{bi}rq^p & \textbf{let } i = 2 \\
          & = q^aq^{2b}rq^p &
    \end{align*}
    
    Since the number of $q$'s in the substring $xy = a+2b$ and the number of $q$'s in the substring $z = p=a+b$, it follows that this language is not regular.
    
    Therefore, $\mathcal{L}$ is not regular.
\end{proof}
